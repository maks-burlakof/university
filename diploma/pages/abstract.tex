\sectioncentered*{Реферат}
\thispagestyle{empty}

СИСТЕМА УПРАВЛЕНИЯ ОФИСНЫМИ ИНФОРМАЦИОННЫМИ ПРОЦЕССАМИ : дипломный проект / М.В. Бурлаков~-- Минск: БГУИР, 2025,~-- п.з.~-- \pageref*{LastPage} с., чертежей~-- 5л. формата А1, плакатов~-- 1л. формата А1. \\

% TODO: проверить сколько плакатов, сколько чертежей

В дипломном проекте решена задача разработки системы управления офисными информационными процессами на предприятии. Выполнен анализ текущих подходов к автоматизации офисных процессов, исследованы существующие программные решения, выявлены их ограничения. Произведено проектирование архитектуры программной системы на основе микросервисного подхода, реализованы модули бронирования рабочих мест, оборудования и помещений, контроля доступа сотрудников, рассадки персонала и уведомлений. Разработаны алгоритмы автоматической рассадки сотрудников, утверждения запроса на бронирование рабочего места. Спроектирована и реализована структура базы данных, отражающая иерархию офисных ресурсов и персонала, а также диаграмма вариантов использования.

Разработано программное обеспечение для решения указанных задач с использованием языка программирования \textit{Python}, фреймворка \textit{FastAPI} для серверной части. Разработан веб-интерфейс для сотрудников и менеджеров с поддержкой авторизации, визуализации данных и отчётности. Для каждого типа пользователя реализованы индивидуальные сценарии взаимодействия с системой. Предусмотрен режим администратора с расширенными возможностями управления пользователями и ресурсами. Подготовлено пользовательское руководство, выполнены функциональные и нагрузочные тестирования компонентов. Проведён технико-экономический анализ, подтверждающий экономическую целесообразность внедрения системы за счёт снижения затрат на администрирование офисных процессов и повышения эффективности использования офисных площадей.
