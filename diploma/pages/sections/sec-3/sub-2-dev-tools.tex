\subsection{Обоснование выбора инструментальных средств разработки}
\label{sub:system-design:dev-tools}

В этом разделе будет произведено рассмотрение основных технологий, использованных при разработке системы.

Для реализации \textit{backend}-архитектуры системы используются следующие технологии:

\begin{itemize}
    \item язык программирования \textit{Python};
    \item веб-фреймворк \textit{FastAPI};
    \item \textit{SQLAlchemy}, \textit{Alembic}, \textit{Pydantic} для взаимодействия с базой данных, выполнения миграций и реализации \textit{ORM}~-- технологии программирования, которая связывает базы данных с концепциями объектно-ориентированных языков программирования, создавая «виртуальную объектную базу данных».
\end{itemize}

Выбор \textit{Python} обусловлен его активным развитием, популярностью в сфере веб-разработки и удобством для работы с асинхронными операциями. \textit{SQLAlchemy} обеспечивает удобное взаимодействие с реляционными базами данных, сокращая количество \textit{SQL}-запросов и повышая читаемость кода.

Реализация клиентской части системы выполнена с использованием следующих технологий:

\begin{itemize}
    \item языков программирования \textit{JavaScript}, \textit{TypeScript};
    \item языков разметки и стилистического оформления \textit{HTML} и \textit{CSS};
    \item библиотеки \textit{React} для разработки пользовательских интерфейсов;
    \item CSS-фреймворков \textit{Tailwind}, \textit{MaterialUI}.
\end{itemize}

Выбор \textit{React} обусловлен наличием компонентного подхода, упрощающего поддержку и масштабируемость кода. \textit{Tailwind CSS} ускоряет верстку за счет готовых классов, а \textit{MaterialUI} предоставляет набор компонентов для обеспечивания удобного и современного пользовательского интерфейса.

В качестве СУБД используется \textit{PostgreSQL} из-за её бесплатного распространения, высокой производительности, надёжности, поддержкой \textit{ACID}-транзакций и расширяемости.

Для развёртывания системы на сервере используется совокупность технологий:

\begin{itemize}
    \item \textit{Docker} для контейнеризации приложений;
    \item \textit{Kubernetes(k3s)} для оркестрации приложений, обеспечивая автоматическое управление нагрузкой, балансировку трафика, восстановление контейнеров в случае сбоя;
\end{itemize}

Системой контроля версий является \textit{Git}, выбранный веб-сервис для хостинга проекта~-- \textit{GitHub}. Технологией непрерывной интеграции/непрерывного развертывания является \textit{GitHub Actions}.


% -------------------------- Убрано
\iffalse

\subsubsection{Язык программирования Python. }

Python~-- это высокоуровневый язык программирования общего назначения. Он является интерпретируемым, то есть его код исполняется непосредственно интерпретатором без предварительной компиляции в машинный код, что позволяет достичь кроссплатформенности, поэтому приложения на Python могут запускаться на различных операционных системах (Windows, macOS, Linux и др.) без необходимости модификации исходного кода. При этом Python автоматически компилирует исходный код в промежуточный байт-код (файлы с расширением .pyc), который затем исполняется интерпретатором. Наиболее популярная реализация Python~-- CPython, написанная на языке C. CPython предоставляет интерпретатор, стандартную библиотеку и возможности для интеграции с модулями на C для повышения производительности.

В программировании активно используются фреймворки~-- структуры, которые предоставляют разработчику готовые инструменты, библиотеки и архитектурные решения для упрощения создания приложений. В экосистеме Python представлено множество фреймворков, охватывающих широкий спектр задач. Например, для веб-разработки существуют Django, Flask, FastAPI и Pyramid, для машинного обучения~-- TensorFlow, PyTorch и Scikit-learn, для тестирования~-- pytest и unittest, а для создания графических интерфейсов~-- PyQt, Tkinter и Kivy.

Веб-фреймворки Python предназначены для упрощения разработки веб-приложений, предоставляя необходимый инструментарий для реализации серверной логики, маршрутизации запросов, работы с базами данных и аутентификации пользователей. FastAPI~-- современный фреймворк, оптимизированный для создания высокопроизводительных API с использованием стандартов OpenAPI и JSON Schema.

Таким образом, Python, благодаря своей интерпретируемости, динамической типизации и богатой экосистеме, является одним из самых популярных языков программирования, а разнообразие фреймворков делает его подходящим для решения практически любых задач в разработке.

\subsubsection{СУБД PostgreSQL. }

PostgreSQL~-- это мощная реляционная система управления базами данных (СУБД) с открытым исходным кодом, которая отличается высокой производительностью, надёжностью и расширяемостью. Она соответствует стандарту SQL:2016 и поддерживает такие продвинутые возможности, как оконные функции, Common Table Expressions (CTE) и полнотекстовый поиск. PostgreSQL обеспечивает выполнение ACID-транзакций, гарантируя целостность данных даже в критических ситуациях, и поддерживает сложные типы данных, включая JSON/JSONB, массивы, XML и пользовательские структуры.

Одним из ключевых преимуществ PostgreSQL является её расширяемость. Разработчики могут создавать собственные типы данных, функции, операторы и индексы, а также использовать множество сторонних расширений, таких как PostGIS для работы с геоданными или TimescaleDB для временных рядов. Система эффективно работает с аналитическими и транзакционными нагрузками, обеспечивая параллельное выполнение запросов, масштабируемость и поддержку репликации (как синхронной, так и асинхронной). Для оптимизации запросов PostgreSQL предоставляет широкий выбор индексов, включая B-Tree, GiST, GIN, BRIN и другие, что делает её универсальной для различных сценариев использования.

По сравнению с другими реляционными СУБД, PostgreSQL обладает рядом преимуществ. Во-первых, это мощная поддержка JSONB, которая позволяет использовать её как гибридную базу данных для работы с документами, конкурируя с NoSQL-системами, такими как MongoDB. Во-вторых, PostgreSQL отличается высокой универсальностью: она одинаково эффективна для транзакционных систем (OLTP) и аналитических систем (OLAP), благодаря поддержке сложных запросов и массивов. В-третьих, система легко масштабируется и поддерживает высокую доступность с помощью таких инструментов, как Patroni и Pgpool-II. Кроме того, благодаря поддержке Foreign Data Wrappers (FDW), PostgreSQL может интегрироваться с другими базами данных и внешними источниками данных~\cite{book_postgres_optimization}.

Особое внимание заслуживает продвинутая система индексации. PostgreSQL позволяет использовать уникальные индексы, такие как GiST и GIN, что делает её идеальной для работы с полнотекстовым поиском и пространственными данными. Благодаря открытому исходному коду PostgreSQL активно развивается и получает регулярные обновления, а её большое сообщество обеспечивает доступ к обширной документации и решениям для специфических задач.

PostgreSQL широко применяется для создания веб-приложений, построения аналитических систем, работы с геоданными и JSON-документами, а также в сценариях, требующих высокой производительности и надёжности. Благодаря своей гибкости, функциональности и поддержке масштабирования, PostgreSQL является одним из лидеров среди современных реляционных СУБД, сочетая в себе возможности классических баз данных и гибкость NoSQL-систем.

\subsubsection{Язык программирования TypeScript и фреймворк React. }

TypeScript и React~-- это современные инструменты, которые часто используются для разработки масштабируемых, производительных и поддерживаемых веб-приложений.

TypeScript представляет собой надмножество JavaScript, добавляющее статическую типизацию и расширенные возможности для разработки, такие как поддержка интерфейсов, перегрузка функций и объединение типов. Код TypeScript компилируется в стандартный JavaScript, что делает его совместимым с любой платформой или браузером. Одним из ключевых преимуществ TypeScript является статическая типизация, которая позволяет выявлять ошибки уже на этапе компиляции.

React~-- это библиотека для построения пользовательских интерфейсов, разработанная Facebook. Она предлагает декларативный стиль программирования, позволяя разработчику описывать, как интерфейс должен выглядеть в зависимости от состояния приложения. Её ключевое преимущество~-- компонентный подход, который позволяет разбивать интерфейс на независимые, многократно используемые компоненты. React использует виртуальный DOM для эффективного обновления только изменённых компонентов вместо полной перерисовки всей страницы~\cite{book_react}.

Таким образом, использование TypeScript в сочетании с React позволяет разработчикам строить современные веб-приложения, которые отличаются высокой надёжностью, эффективностью и удобством поддержки. TypeScript обеспечивает строгую типизацию и улучшенный контроль кода, а React предлагает гибкость и мощные инструменты для создания интерфейсов, делая их сочетание особенно привлекательным для сложных и долгосрочных проектов.

\subsubsection{Технологии развёртывания. }

Docker и Kubernetes~-- это ключевые технологии современной разработки, которые обеспечивают контейнеризацию и оркестрацию приложений. Они играют важную роль в упрощении развертывания, управления и масштабирования программных решений.

Контейнеризация представляет собой метод упаковки приложений и всех их зависимостей в изолированные среды, называемые контейнерами. Эти контейнеры используют общее ядро операционной системы, что делает их лёгкими, производительными и удобными для переноса между различными средами. Основные преимущества контейнеризации включают портативность (контейнеры работают одинаково на любых платформах), эффективное использование ресурсов (благодаря отсутствию необходимости эмулировать полноценную ОС, как в виртуальных машинах), изоляцию приложений для повышения безопасности, а также упрощение разработки и развертывания за счёт стандартизированных сред.

Docker~-- это платформа, предназначенная для создания, развертывания и управления контейнерами. С его помощью разработчики могут создавать образы контейнеров с использованием Dockerfile, управлять ими через Docker Registry, а также запускать контейнеры с минимальными накладными расходами. Docker предоставляет возможности для работы с сетями, хранения данных и автоматизации жизненного цикла контейнеров.

Однако для управления множеством контейнеров в масштабах кластера Docker нуждается в системе оркестрации, что достигается с помощью Kubernetes. Kubernetes~-- это открытая система оркестрации, которая автоматизирует развертывание, управление и масштабирование контейнеризированных приложений. Она изначально была разработана Google и предоставляет богатый функционал для управления контейнерными кластерами.

Одним из ключевых преимуществ Kubernetes является автоматическое масштабирование и репликация приложений. С помощью механизма ReplicaSet система гарантирует, что всегда будет поддерживаться заданное количество копий контейнеров, что обеспечивает надёжность и устойчивость к сбоям. Kubernetes также упрощает сетевое взаимодействие: компонент Ingress позволяет маршрутизировать входящие запросы к контейнерам через HTTP/HTTPS, настраивать балансировку нагрузки и SSL-шифрование. Кроме того, поддержка сетевых политик позволяет разработчикам контролировать, как трафик проходит между компонентами приложения.

Управление состоянием приложений в Kubernetes реализовано через Persistent Volume (PV) и Persistent Volume Claim (PVC), которые обеспечивают удобное хранение данных, даже если контейнеры перезапускаются~\cite{book_production_kubernetes}. Система поддерживает обновления приложений без простоев (rolling updates) и откаты на предыдущие версии, если в процессе развертывания возникают проблемы. Kubernetes также обеспечивает высокую доступность приложений благодаря встроенным механизмам отказоустойчивости, включая мониторинг состояния контейнеров через Liveness и Readiness Probes.

Ещё одно значимое преимущество Kubernetes~-- это его расширяемость. Пользовательские ресурсы (Custom Resource Definitions, CRD) позволяют добавлять новые типы объектов, а богатая экосистема включает инструменты для мониторинга (Prometheus), журналирования (ELK), автоматизации CI/CD (ArgoCD) и многое другое. Kubernetes также предоставляет встроенные механизмы для управления конфигурациями (ConfigMaps) и секретами (Secrets), что упрощает хранение и использование конфиденциальных данных.

Вместе Docker и Kubernetes создают мощную платформу для разработки, развертывания и эксплуатации современных приложений. Docker упрощает процесс контейнеризации, а Kubernetes обеспечивает эффективное управление этими контейнерами на уровне кластера, предоставляя инструменты для репликации, отказоустойчивости, маршрутизации через Ingress и управления состоянием. Эти технологии формируют основу облачной инфраструктуры, поддерживая автоматизацию, масштабируемость и высокую доступность приложений.

\subsubsection{Технологии непрерывной интеграции/непрерывного развертывания. }

CI/CD (Continuous Integration/Continuous Deployment)~-- это подход и набор инструментов, направленных на автоматизацию ключевых этапов разработки программного обеспечения, включая тестирование, сборку и развертывание. Эти методы помогают ускорить процесс разработки, повысить качество кода и уменьшить вероятность ошибок, делая проект более стабильным и готовым к частым обновлениям.

Continuous Integration (CI) подразумевает регулярное объединение изменений в коде с основной веткой репозитория. Каждый коммит или запрос на слияние запускает автоматизированный процесс проверки, включающий тестирование и сборку приложения. Цель CI~-- выявлять ошибки на ранних стадиях разработки, предотвращая проблемы, которые могут возникнуть при интеграции изменений.

Continuous Deployment (CD) автоматизирует процесс доставки изменений в различные среды~-- тестовую, промежуточную или к конечным пользователям. В рамках подхода Continuous Delivery развертывание требует подтверждения человека, в то время как в Continuous Deployment все изменения, успешно прошедшие тесты, автоматически развёртываются. Это сокращает время выхода обновлений и минимизирует человеческий фактор.

При разработке данной системы использовался GitHub Actions - инструмент для автоматизации CI/CD, встроенный в экосистему GitHub. Он предоставляет возможность автоматического запуска процессов в ответ на события в репозитории, такие как внесение изменений, создание запросов на слияние или выпуск новой версии.

\fi
% -------------------------- Убрано
