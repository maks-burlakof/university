\section{Реализация системы управления офисными информационными процессами}
\label{sec:system-implementation}

\subsection{Организация аппаратного комплекса системы}
\label{sub:system-implementation:hardware}

\FPeval{\NumOfWorkstations}{134}

В соответствии с расположением рабочих станций в отделах и техническими требованиями к системе, а также выбранной клиент-серверной архитектурой разработан аппаратно-технический комплекс системы.

В состав аппаратного-технического комплекса входят:

\begin{itemize}
    \item рабочие места, представленные в виде~\NumOfWorkstations~стационарных компьютеров, подключенных по локальной сети предприятия с помощью модема и коммутаторов;
    \item сервер со специальным программным обеспечением.
\end{itemize}

Аппаратная платформа сервера оптимизирована для нагрузки в 1000~-- 2000 запросов в минуту с упором на экономическую эффективность. Основу серверной инфраструктуры составляют процессоры среднего класса, поддерживающие аппаратную виртуализацию. Подробные требования к аппаратной платформе приведены в таблице~\ref{table:system-implementation:hardware:requirements}.

\begingroup
\singlespacing
\vspace{-\baselineskip}
\begin{longtable}{|>{\raggedright}m{0.2\textwidth}
                  |>{\raggedright}m{0.35\textwidth}
                  |>{\raggedright\arraybackslash}m{0.37\textwidth}|}
    \caption{Требования к аппаратной платформе} \label{table:system-implementation:hardware:requirements} \\ \hline
    Компонент & Минимальные характеристики & Рекомендуемые характеристики \\ \hline
    \endfirsthead
    \multicolumn{3}{@{}l}{\noindent Продолжение таблицы~\thetable} \\ \hline
    Компонент & Минимальные характеристики & Рекомендуемые характеристики \\ \hline
    \endhead
    Процессор & 4 ядра (\textit{Intel Core i5} / \textit{AMD Ryzen 5}) & 8 ядер (\textit{Intel Core i7} / \textit{AMD Ryzen 7}) \\
    \hline
    ОЗУ & 16 ГБ DDR4 & 32 ГБ DDR4 (16 ГБ для \textit{PostgreSQL}) \\
    \hline
    Хранилище & 500 ГБ \textit{SATA} SSD (чтение/запись: 550 МБ/с) & 1 ТБ \textit{NVMe} SSD (чтение/запись: 3500 МБ/с, RAID 1) \\
    \hline
    \textit{Ethernet} & 1 Гбит \textit{Ethernet} & 2×1 Гбит (LACP для балансировки) \\
    \hline
    Видео выход & Интегрированная графика (на базе CPU) & Базовая дискретная видеокарта (для управления) \\
    \hline
    USB порты & 2×USB 3.0 (для периферии/аппаратных ключей) & 4×USB 3.0 (с поддержкой шифрованных носителей) \\
    \hline
    Питание & Блок питания \textit{80+ Bronze} & Блок питания \textit{80+ Platinum} + ИБП 1500VA \\
    \hline
    Уровень шума & <35 дБ (офисное размещение) & <25 дБ (для шумочувствительных зон) \\
    \hline
    Безопасность & TPM 2.0 (опционально) & Аппаратный TPM + поддержка \textit{Secure Boot} \\
    \hline
    Резервное копирование & Внешний NAS (\textit{rsync}) & Ленточный накопитель или облачная \textit{S3}-репликация \\
    \hline
    Поддержи\-ва\-емые ОС & \textit{Ubuntu Server 22.04 LTS} & \textit{RHEL 9} / \textit{Rocky Linux 9} / специализированные дистрибутивы для \textit{Kubernetes} (\textit{k3OS}, \textit{RancherOS}) \\
    \hline
\end{longtable}
\endgroup

Сетевая инфраструктура офисной информационной системы построена с использованием современного аппаратного обеспечения, обеспечивающего стабильную, безопасную и высокоскоростную передачу данных внутри организации. В основе проводной сети лежат управляемые коммутаторы, установленные в распределительном шкафу. Они обеспечивают коммутацию трафика между сегментами сети и поддерживают функции приоритезации трафика (\textit{QoS}), виртуальных локальных сетей (\textit{VLAN}), а также зеркалирования портов для сетевого мониторинга. Все порты коммутаторов рассчитаны на гигабитную скорость передачи данных, что обеспечивает необходимую пропускную способность для офисных приложений.

Маршрутизация и управление доступом к внешним сетям реализованы с помощью выделенного маршрутизатора. Устройство поддерживает механизмы трансляции сетевых адресов (\textit{NAT}), статической и динамической маршрутизации, а также базовые функции межсетевого экранирования. Это обеспечивает надёжное соединение с интернетом и базовую защиту внутренней сети~\cite{book_tanenbaum_computer_networks}.

Для организации беспроводного доступа по всей территории офиса используются точки доступа, подключённые к той же проводной инфраструктуре. Они поддерживают одновременную работу в нескольких диапазонах частот, обеспечивая равномерное покрытие и устойчивую работу мобильных и переносных устройств.

Кабельная система построена на основе экранированной витой пары категории 6, с соблюдением стандартов структурированной кабельной системы. Все коммуникации проложены в скрытых кабель-каналах с выходом в настенные розетки, что обеспечивает как надёжность соединения, так и удобство эксплуатации.

Инфраструктура включает также оборудование для распределения электропитания (\textit{Patch}-панели, стойки, блоки питания с грозозащитой) и систему резервного питания, обеспечивающую бесперебойную работу ключевых сетевых компонентов в случае отключения электроэнергии.

На рис.~\ref{fig:system-implementation:hardware:logical-network} приведена схема логической структуры сети, отражающая принципы организации взаимодействия между основными элементами информационной системы: пользовательскими рабочими станциями, серверными компонентами, сетевым оборудованием и внешними сервисами. 

\begin{figure}[h]
\centering
    \includegraphics[width=0.8\linewidth]{assets/hardware-logical-network-structure.png}
    \caption{Схема логической структуры сети}
    \label{fig:system-implementation:hardware:logical-network}
\end{figure}

Пользовательский сегмент состоит из офисных рабочих станций, ноутбуков и мобильных устройств сотрудников, подключённых через \textit{Ethernet} или \textit{Wi-Fi}. Все устройства получают IP-адреса по \textit{DHCP} от внутреннего сервера или маршрутизатора. Доступ пользователей к внутренним и внешним ресурсам ограничен политиками безопасности на уровне сетевого шлюза и межсетевого экрана.

Точки доступа \textit{Wi-Fi} (\textit{Access Points}) обеспечивают подключение мобильных устройств и ноутбуков. Поддержка \textit{WPA2/WPA3}, сегментация по \textit{VLAN} и ограничение гостевого доступа применяются для повышения безопасности.

Управляемые коммутаторы (\textit{L2/L3}) объединяют все конечные устройства и серверы, обеспечивая передачу данных по внутренней сети. \textit{VLAN} используются для логического разделения трафика (например, пользователи, серверы, \textit{VoIP}).

Серверный сегмент включает в себя:

\begin{itemize}
    \item веб-сервер, который обслуживает \textit{HTTP/HTTPS}-запросы к системе;
    \item сервер приложений, исполняющий бизнес-логику;
    \item базу данных~-- \textit{PostgreSQL}, работающую в защищённом сегменте без прямого доступа из внешней сети;
    \item файловые хранилища — для обмена и резервного копирования.
\end{itemize}

Между компонентами применяются закрытые соединения по \textit{TCP/IP}, защищённые \textit{TLS} или локальной сетью без выхода наружу.

Внешний трафик, включая обращения к \textit{API} сторонних систем, обновления ПО, а также почтовые и облачные сервисы, направляется через маршрутизатор с \textit{TLS}-шифрованием. \textit{DNS}-запросы и время синхронизируются с доверенными внешними серверами.

Хранилище данных организовано на локальных \textit{SSD} с ежедневным резервным копированием на внешние \textit{NAS} (например, \textit{Synology DS220+}) через \textit{rsync}. Репликация \textit{PostgreSQL} настроена в режиме \textit{master-standby} с асинхронной синхронизацией, что исключает простои при сбоях. Для \textit{Kubernetes Persistent Volumes} используются локальные тома (\textit{LocalPV}), что упрощает управление и снижает затраты на распределенные системы.

Энергоэффективность достигается за счет серверов с блоками питания \textit{80+ Bronze} и настройкой режимов энергосбережения \textit{CPU}. Рекомендуется использование резервного блока питания для защиты от перебоев. Для безопасности используются:

\begin{itemize}
    \item базовый фаервол (\textit{iptables}/\textit{nftables}) с ограничением доступа по \textit{IP} или \textit{MAC}-адресам;
    \item бесплатные \textit{TLS}-сертификаты (\textit{Let’s Encrypt}) для шифрования трафика;
    \item регулярные обновления ОС и ПО через встроенные менеджеры пакетов.
\end{itemize}

Такая конфигурация обеспечивает задержку API-запросов <100 мс. Использование открытого ПО (\textit{Kubernetes}, \textit{PostgreSQL}, \textit{HAProxy}) и стандартного оборудования минимизирует лицензионные расходы, сохраняя гибкость для будущих модернизаций.

Транспортный уровень системы реализован на базе стандартных протоколов, оптимизированных для умеренных нагрузок (1000–2000 \textit{RPM}) и локальной инфраструктуры предприятия. Схема аппаратно-технического комплекса системы представлена на рис.~\ref{fig:system-implementation:hardware:hardware-complex}.

\afterpage{
    \clearpage
    \begin{landscape}
        \thispagestyle{landscape}
        \begin{figure}[p]
            \centering
            \includegraphics[width=0.99\linewidth]{assets/hardware-complex.png}
            \caption{Схема аппаратно-технического комплекса системы}
            \label{fig:system-implementation:hardware:hardware-complex}
        \end{figure}
    \end{landscape}
    \clearpage
}

\textit{\textbf{Протокол TCP (Transmission Control Protocol)}} обеспечивает надёжную, ориентированную на соединение передачу данных между компонентами системы, что критично для синхронных операций, таких как \textit{HTTP}-запросы к \textit{REST API}, взаимодействие с базой данных PostgreSQL. На уровне установления соединения используется трёхэтапное рукопожатие (\textit{SYN, SYN-ACK, ACK}), гарантирующее согласованность параметров обмена.

После установления соединения \textit{TCP} реализует потоковую передачу данных, разбивая сообщения на сегменты. Каждый сегмент снабжается порядковым номером, что позволяет принимающей стороне упорядочивать полученные данные и обнаруживать пропущенные сегменты. Протокол требует подтверждения получения (\textit{ACK}) для каждого сегмента, и в случае отсутствия подтверждения инициирует повторную передачу, обеспечивая таким образом надёжность доставки.

Протокол \textit{TCP} также включает механизмы управления потоком (\textit{flow control}), основанные на размере окна приёма (\textit{window size}), что позволяет получателю регулировать скорость поступления данных в зависимости от своей обработки. Дополнительно реализовано управление перегрузкой (\textit{congestion control}), позволяющее адаптировать скорость передачи к текущей загрузке сети. На практике это достигается с помощью алгоритмов, таких как \textit{Reno}, \textit{NewReno} или \textit{Cubic}, которые динамически увеличивают или уменьшают размер окна передачи, предотвращая потери данных при перегрузке сети.

Для рассматриваемой системы с нагрузкой 1000–2000 \textit{RPM} ключевыми преимуществами \textit{TCP} являются встроенный механизм повторной передачи потерянных пакетов, автоматическая коррекция порядка доставки и адаптивное управление трафиком, что минимизирует риски потери данных в условиях нестабильного сетевого канала и гарантирует согласованную работу распределённых компонентов.

\textit{\textbf{Протокол AMQP (Advanced Message Quepping Protocol)}} обеспечивает асинхронную коммуникацию между микросервисами через брокер сообщений \textit{RabbitMQ}, реализуя событийно-ориентированную архитектуру рассматриваемой системы. В условиях локальной инфраструктуры \textit{AMQP} работает поверх \textit{TCP}. Это важно для соблюдения гарантий: \textit{TCP} устраняет потерю пакетов, а \textit{AMQP} добавляет уровень логической надежности.

\textit{\textbf{Протокол TLS (Transport Layer Security)}}~-- криптографический протокол, лежащий в основе HTTPS. Процесс шифрования запроса начинается с обмена уникальными сессионными ключами для каждого соединения, что исключает компрометацию прошлых сессий при утечке ключей. После проверки сертификата данные шифруются симметричными алгоритмами. Все внешние \textit{HTTP}-запросы перенаправляются на \textit{HTTPS} с использованием \textit{TLS} 1.3 и сертификатов \textit{Let’s Encrypt}, что защищает данные от перехвата.

Технически, протокол \textit{TLS} работает поверх \textit{TCP} и обеспечивает целостность и конфиденциальность передаваемой информации. Установление соединения происходит через процесс \textit{TLS Handshake}, в ходе которого стороны обмениваются криптографическими параметрами, согласуют алгоритмы шифрования (\textit{cipher suites}) и проводят проверку подлинности с помощью \textit{X.509}-сертификатов. Начиная с версии \textit{TLS} 1.3, протокол существенно упростился: большинство уязвимых и устаревших алгоритмов исключено, \textit{handshake} стал короче (обычно одна \textit{RTT}), а поддержка прямой конфиденциальности (\textit{forward secrecy}) гарантирована использованием протокола обмена ключами \textit{Diffie–Hellman (ECDHE)}.

Шифрование трафика осуществляется симметричными алгоритмами, обеспечивающими высокую скорость работы и стойкость к атаке с подбором. Дополнительно используется \textit{HMAC} (или \textit{AEAD}) для проверки целостности сообщений, предотвращая любые несанкционированные модификации данных в процессе передачи.

На рис.~\ref{fig:system-implementation:hardware:tls-layers-integration} представлена схема слоевой интеграции \textit{TLS} в сетевую модель, иллюстрирующая работу \textit{TLS} в контексте сетевого взаимодействия. \textit{TLS} работает между прикладным и транспортным уровнями: оно шифрует и расшифровывает данные, передаваемые между приложением и \textit{TCP}. \textit{TLS} шифрует только данные приложения. Заголовки \textit{TCP/IP} остаются в открытом виде, чтобы маршрутизаторы могли доставить пакет.

\begin{figure}[h]
\centering
    \includegraphics[width=0.7\linewidth]{assets/tls-layers-integration.png}
    \caption{Процесс интеграции \textit{TLS} в сетевую модель взаимодействия}
    \label{fig:system-implementation:hardware:tls-layers-integration}
\end{figure}

\textit{\textbf{Протокол SSH (Secure Shell)}}~-- это криптографический протокол, предназначенный для безопасного удалённого управления операционными системами и передачи данных через незащищённые сети. Работая поверх \textit{TCP} (порт 22 по умолчанию), он обеспечивает полное шифрование трафика, аутентификацию и целостность данных~\cite{book_olifer_network_os}. Используя \textit{SSH} выполняется настройка сервера, конфигурация всех компонентов системы, обновление программного комплекса системы, мониторинг производительности и тд.


\subsection{Организация программного комплекса системы}
\label{sub:system-implementation:backend-architecture}

Основной целью в реализации программной части системы является чёткое разделение задач, что достигается следованием принципам чистой архитектуры. Это достигается путем разделения программного обеспечения на слои, где существует как минимум один слой для бизнес-правил, а другой~-- для интерфейсов. На сегодняшний это одно из самых популярных решений для разработки, так как обеспечивает читаемость, расширяемость, устойчивость и гибкость. Чистая архитектура помогает избежать проблем с зависимостями и разделить структуру программного комплекса на логические блоки. Ниже приведены основные преимущества системы, использующей такой подход.

\begin{itemize}
    \item независимость от фреймворков. Архитектура не зависит от существования некоторой библиотеки программного обеспечения с множеством функций. Это позволяет использовать такие фреймворки как инструменты, а не встраивать свою систему в их ограниченные рамки;
    \item тестируемость. Бизнес-правила можно тестировать без пользовательского интерфейса, базы данных, веб-сервера или любого другого внешнего элемента;
    \item независимость от пользовательского интерфейса. Пользовательский интерфейс можно легко изменить, не меняя остальную часть системы. Например, веб-интерфейс может быть заменен на консольный, без изменения бизнес-правил;
    \item независимость от базы данных, что позволяет заменить \textit{Oracle} или \textit{SQL Server} на \textit{MongoDB}, \textit{PostgreSQL}, \textit{MySQL} или что-то еще. Бизнес-правила не привязаны к базе данных.
\end{itemize}

Главный принцип чистой архитектуры~-- разделение приложения на уровни, каждый из которых выполняет свои задачи и является независимым от других слоев. Для реализации такого поведения необходимо придерживаться правила зависимостей. Диаграмма на рис.~\ref{fig:clean-architecture-diagram} наглядно отображает взаимосвязь слоёв приложения. Каждый из них представляет различные области программного обеспечения. По мере продвижения вглубь программное обеспечение становится все более абстрактным и включает в себя политики более высокого уровня. Самый внутренний круг~-- самый общий. Внешние круги~-- это механизмы и конкретные реализации. Внутренние круги~-- это правила, ограничения и политики~\cite{web_clean_architecture}.

\begin{figure}[h]
\centering
    \includegraphics[width=0.9\linewidth]{assets/clean-architecture-diagram.png}
    \caption{Диаграмма взаимосвязи слоёв приложения}
    \label{fig:clean-architecture-diagram}
\end{figure}

\subsubsection{Правило зависимостей. }

Главным правилом, благодаря которому эта архитектура работает, является правило зависимостей (\textit{The Dependency Rule}). Это правило гласит, что зависимости исходного кода могут быть направлены только внутрь. Ничто во внутреннем круге не может знать что-либо о чем-то во внешнем круге. В частности, имя чего-то, объявленного во внешнем круге, не должно упоминаться в коде внутреннего круга. Это относится к функциям, классам, переменным и любым другим именованным программным объектам.

Таким же образом форматы данных, используемые во внешнем круге, не должны использоваться во внутреннем круге, особенно если эти форматы генерируются фреймворком во внешнем круге.

\subsubsection{Сущности. }

Сущности инкапсулируют бизнес-правила для всего предприятия. Сущность может быть объектом с методами, а может быть набором структур данных и функций. Это не имеет значения, если сущности могут использоваться различными приложениями на предприятии. Они с наименьшей вероятностью будут меняться при изменении чего-то внешнего. Никакие операционные изменения в конкретном приложении не должны влиять на слой сущностей. В листинге ниже представлена модель сущности сотрудника предприятия, реализованная с помощью класса данных (\textit{Dataclass}), предоставляемого встроенным в \textit{Python} модулем \textit{dataclasses}:

\begin{lstlisting}[style=pythonstyle]
class EmployeeRoleOption(StrEnum):
    employee = "EMPLOYEE"
    office_manager = "OFFICE_MANAGER"
    facility_manager = "FACILITY_MANAGER"
    it_support = "IT_SUPPORT"
    receptionist = "RECEPTIONIST"
    safety_officer = "SAFETY_OFFICER"
    hr = "HR"
    resource_manager = "RESOURCE_MANAGER"
    admin = "ADMIN"

@dataclass
class Employee:
    first_name: str
    last_name: str
    email: str
    department_id: UUID
    position_id: UUID
    professional_level_id: UUID
    phone_number: Optional[str] = None
    role: EmployeeRoleOption = EmployeeRoleOption.employee
    middle_name: Optional[str] = None
    picture: Optional[str] = None
    is_blocked: bool = False
    is_archived: bool = False
    created_at: datetime = field(default_factory=datetime.now)
    modified_at: datetime = field(default_factory=datetime.now)
    id: UUID = field(default_factory=uuid4)
\end{lstlisting}

Атрибут \textit{role} для модели сотрудника является объектом класса \textit{Employee\-Role\-Option}, наследованным от \textit{StrEnum}. Стоит отметить, что \textit{Dataclass} в \textit{Python} не предоставляет возможностей для проверки правильности с точки зрения бизнес-логики атрибутов класса, без явного переопределения метода \textit{\lstinline!__post_init__!}. Как упоминалось ранее, операции по проверке осуществляются во внешних слоях (интерфейсами и контроллерами).

\subsubsection{Сценарии использования. }

Программное обеспечение на этом уровне содержит бизнес-правила, специфичные для конкретного приложения. Оно инкапсулирует и реализует все сценарии использования системы, организуя поток данных к сущностям и от них, а также направляют эти сущности на использование бизнес-правил в масштабах предприятия для достижения целей сценария использования.

При этом не ожидается, что изменения в этом слое повлияют на сущности или что на этот слой повлияют изменения внешних факторов, таких как база данных, пользовательский интерфейс или любой из общих фреймворков. Этот слой изолирован от подобных проблем.

Тем не менее, ожидается, что изменения в работе приложения повлияют на сценарии использования и, следовательно, на программное обеспечение в этом слое. Если детали сценария использования изменятся, то код в этом слое обязательно будет затронут. Пример реализованного для рассматриваемой системы сценария использования приведен в следующем листинге:

\begin{lstlisting}[style=pythonstyle]
class ApproveWorkspaceOccupationRequestUseCase:
    def __init__(
        self,
        employee_repository: EmployeeRepository,
        workspace_repository: WorkspaceRepository,
        workspace_occupation_repository: WorkspaceOccupationRepository,
        occupation_request_repository: OccupationRequestRepository,
    ):
        self._employee_repository = employee_repository
        self._workspace_repository = workspace_repository
        self._workspace_occupation_repository = workspace_occupation_repository
        self._occupation_request_repository = occupation_request_repository

    async def __call__(
        self, request_id: UUID, approver_employee: Employee, approver_comment: str
    ) -> None:
        occupation_request = await self._occupation_request_repository.get(
            id=request_id
        )
        await self._validate_occupation_request(occupation_request)

        requested_workspace = await self._workspace_repository.get(
            id=occupation_request.object_id
        )
        self._validate_workspace(requested_workspace)

        occupation = WorkspaceOccupation(
            type=occupation_request.occupation_type,
            workspace_id=occupation_request.object_id,
            employee_id=occupation_request.requester_employee_id,
            start_date=occupation_request.start_date,
            end_date=occupation_request.end_date,
        )
        occupation = await self._workspace_occupation_repository.create(occupation)
        
        occupation_request.status = OccupationRequestStatusOption.approved
        occupation_request.approver_employee_id = approver_employee.id
        occupation_request.approver_comment = approver_comment
        occupation_request.modified_at = datetime.now()
        await self._occupation_request_repository.update(occupation_request)

        await self._workspace_repository.update(
            id=occupation_request.object_id,
            status=WorkspaceStatusOption.occupied,
            modified_at=datetime.now(),
        )

    async def _validate_occupation_request(
        self, occupation_request: OccupationRequest
    ) -> None:
        if occupation_request.is_archived:
            raise OccupationRequestArchivedError
        if occupation_request.status != OccupationRequestStatusOption.in_progress:
            raise OccupationRequestInvalidStatusError(
                f"Occupation request status is {occupation_request.status}, expected IN_PROGRESS"
            )
        if occupation_request.request_type != OccupationRequestTypeOption.workspace:
            raise OccupationRequestInvalidTypeError(
                f"Occupation request type is {occupation_request.request_type}, expected WORKSPACE"
            )
        if occupation_request.start_date > occupation_request.end_date:
            raise OccupationRequestInvalidDateError(
                f"Occupation request start date {occupation_request.start_date} is after end date {occupation_request.end_date}"
            )

        existing_occupation = await self._workspace_occupation_repository.get_current()
        if existing_occupation:
            raise OccupationRequestAlreadyOccupiedError(
                "There is already an existing occupation for this workspace"
            )

        requester_employee = await self._employee_repository.get(
            id=occupation_request.requester_employee_id
        )
        if requester_employee.is_archived:
            raise EmployeeArchivedError(f"Requested employee is archived")
        if requester_employee.is_blocked:
            raise EmployeeBlockedError(f"Requested employee is blocked")

    def _validate_workspace(self, workspace: Workspace) -> None:
        if not workspace:
            raise WorkspaceDoesNotExistError(
                f"Requested workspace with ID {occupation_request.object_id} does not exist"
            )
        if workspace.is_archived:
            raise WorkspaceArchivedError
        if workspace.status != WorkspaceStatusOption.free:
            raise WorkspaceInvalidStatusError(f"Requested workspace is not free")
\end{lstlisting}

Класс \textit{ApproveWorkspaceOccupationRequestUseCase}, содержащий реализацию конкретной части бизнес-логики, содержит атрибуты, являющиеся объектами конкретных реализаций контроллеров, в данном случае~-- репозитории сущностей. Для аннотации типов используются абстрактные классы соответствующих адаптеров. В качестве примера рассмотрим репозиторий сотрудников, где \textit{EmployeeRepository}~-- абстрактный класс, содержащий определения основных методов для работы с объектами типа \textit{Employee}, а класс \textit{PostgresEmployeeRepository} является конкретной его реализацией для базы данных \textit{PostgreSQL}. Для юнит-тестирования принято также иметь классы для хранения объектов в памяти, например, для рассмотренного случая класс \textit{InMemoryEmployeeRepository} реализует эту возможность.

Таким образом, принципы объектно-ориентированного программирования (ООП) обеспечивают гибкость, модульность и тестируемость системы. Бизнес-логика сценария использования инкапсулирована внутри класса. Внешние зависимости передаются через конструктор, что скрывает детали их реализации от основного кода.

Использование абстрактных классов позволяет отделить определение от конкретной реализации, что соответствует принципу зависимости от абстракций (\textit{DIP}). Класс \textit{ApproveWorkspaceOccupationRequestUseCase} отвечает только за утверждение запросов, а репозитории — за работу с даннымы, что соответствует принципу единственной ответственности (\textit{SRP}). Система открыта для расширения (можно добавить новые репозитории для других СУБД), но закрыта для модификации, что соответствует принципу открытости/закрытости (\textit{OCP}).

Пример с репозиториями иллюстрирует, как абстракции и инъекция зависимостей делают систему адаптируемой к изменениям инфраструктуры (аппаратной или программной).

\subsubsection{Адаптеры интерфейсов. }

Программное обеспечение этого слоя представляет собой набор адаптеров, которые преобразуют данные из формата, наиболее удобного для сценариев использования и сущностей, в формат, наиболее удобный для какого-либо внешнего слоя, например базы данных или \textit{Web}. Представления и контроллеры~-- все они находятся здесь. Модели~-- это просто структуры данных, которые передаются от контроллеров к сценариям использования, а затем обратно от сценариев использования к представлениям~\cite{book_clean_architecture}. Этот слой реализует паттерн проектирования \textit{Adapter}, обеспечивая взаимодействие между абстрактной бизнес-логикой и конкретными технологическими решениями.

Аналогично, в этом слое данные преобразуются из формы, наиболее удобной для сущностей и сценариев использования, в форму, наиболее удобную для любого используемого фреймворка или базы данных. Кроме того, данный слой обычно отвечает за проверку и сериализацию входящих и исходящих данных. В нашем случае эта функциональность делегирована \textit{FastAPI} (в частности, \textit{Pydantic}). Для разрешения зависимостей используется встроенный в \textit{FastAPI} механизм инъекции зависимостей~\cite{book_lubanovich_fastapi} (\textit{Dependency injection}).

Для рассмотренного ранее примера, различные репозитории для управления данными являются адаптерами в контексте архитектуры. Кроме того, адаптерами также могут выступать различные инструменты для интеграции сторонних систем, например для контроля посещаемости, или инструменты аутентификации. Схема взаимосвязи компонентов для рассмотренного примера утверждения запроса на бронирование представлена на рис.~\ref{fig:system-implementation:software:clean-architecture-adapter-components}.

\begin{figure}[h]
\centering
    \includegraphics[width=0.9\linewidth]{assets/clean-architecture-adapter-components.png}
    \caption{Схема взаимосвязи компонентов для процесса утверждения запроса на бронирование}
    \label{fig:system-implementation:software:clean-architecture-adapter-components}
\end{figure}

Адаптеры выполняют три ключевые функции:

\begin{enumerate}
    \item Трансляция данных: преобразование структур данных между внутренним (доменным) и внешним (техническим) представлением. Например, объект сущности \textit{Employee} сериализуется в \textit{JSON} для \textit{REST API} или в строку \textit{SQL} для \textit{PostgreSQL}.
    \item Интеграция с внешними системами: реализация протоколов взаимодействия с базами данных, сторонними \textit{API}, инструментами аутентификации (\textit{OAuth2}, \textit{JWT}) или механизмами кэширования (\textit{Redis}).
    \item Изоляция изменений: сокрытие деталей работы внешних компонентов от бизнес-логики. Замена базы данных, к примеру, потребует лишь создания нового адаптера без модификации кода сценария использования.
\end{enumerate}

В системе можно выделить несколько категорий адаптеров:

\begin{itemize}
    \item контроллеры \textit{API}, например, \textit{FastAPI}-роутеры выступают адаптерами для \textit{HTTP}. Они принимают запросы и преобразуют их в \textit{DTO} (\textit{Data Transfer Objects}) с помощью \textit{Pydantic}, вызывают сценарии использования, передавая параметры в доменном формате и сериализуют результаты в ответы \textit{API};
    \item интеграционные адаптеры, которые являются классами для работы с внешними сервисами, например адаптер для отправки уведомлений может поддерживать как \textit{SMTP}, так и \textit{Slack}-каналы;
    \item репозитории, являющиеся классами вроде \textit{Postgres\-Employee\-Re\-po\-si\-to\-ry}, как показано в листинге ниже, которые реализуют интерфейс \textit{Employee\-Re\-po\-si\-to\-ry}, абстрагируя доступ к данным. Они инкапсулируют \textit{SQL}-запросы, \textit{ORM}-модели и транзакции.
\end{itemize}

\begin{lstlisting}[style=pythonstyle]
class EmployeeRepository(ABC):
    @abstractmethod
    async def get(self, **filters: Any) -> Employee | None:
        pass

    @abstractmethod
    async def get_many(
        self,
        filters: dict | None = None,
        or_filters: dict | None = None,
        or_ilike_filters: dict | None = None,
        sort_by: (Literal["id", "first_name", "last_name", "email", "phone_number", "created_at", "modified_at"] | None) = "id",
        order_by: Literal["asc", "desc"] | None = "asc",
        page: int = 1,
        limit: int = 100,
    ) -> tuple[int, list[Employee]]:
        pass

    @abstractmethod
    async def create(self, employee: Employee) -> Employee:
        pass

    @abstractmethod
    async def update(self, updated_user: Employee, **filters: Any) -> Employee | None:
        pass

    @abstractmethod
    async def delete(self, **filters) -> None:
        pass
\end{lstlisting}

Для оптимизации работы с СУБД \textit{PostgreSQL} используются такие подходы как: пулы соединений (\textit{asyncpg.Pool}) для снижения накладных расходов, выполнение асинхронных запросов через \textit{SQLAlchemy Core}, избегая блокировки \textit{event loop} и индексирование таблиц по часто используемым в выборках полям. Пример запроса получения данных сотрудника из базы данных \textit{PostgreSQL}, используя \textit{SQLAlchemy}, представлен в листинге ниже:

\begin{lstlisting}[style=pythonstyle]
async def get(self, _or: bool = False, **filters: Any) -> User | None:
    try:
        query = select(UserModel)
        if _or:
            query = query.filter(or_(*[getattr(UserModel, key) == value for key, value in filters.items()]))
        else:
            query = query.filter_by(**filters)
        user = await self.session.execute(query)
        user = user.scalars().first()
        return self.__to_user_entity(user) if user else None
    except Exception as e:
        raise DatabaseError(e)
\end{lstlisting}

Интеграция с \textit{RabbitMQ} выполнена с использованием адаптера для брокера сообщений. Его задачами являются: сериализация доменных события (например, \textit{WorkspaceOccupiedEvent}) в \textit{JSON}, отправка сообщений в очередь \textit{\lstinline!workspace_events!}, обработка подтверждения доставки и повторные попытки. В листинге ниже представлен класс \textit{RabbitMQEventPublisher} для публикации событий в очередь \textit{RabbitMQ}:

\begin{lstlisting}[style=pythonstyle]
class RabbitMQEventPublisher:
    def __init__(self, channel: aio_pika.Channel):
        self._channel = channel

    async def publish(self, event: DomainEvent) -> None:
        message = aio_pika.Message(
            body=event.json().encode(),
            headers={"event_type": event.__class__.__name__}
        )
        await self._channel.default_exchange.publish(
            message, routing_key="workspace_events"
        )
\end{lstlisting}

Таким образом, адаптеры интерфейсов служат «мостом» между стабильным ядром системы и изменчивым внешним окружением.

\subsubsection{Фреймворки и драйверы. }

Внешний слой системы представляет собой инфраструктурные компоненты, которые обеспечивают взаимодействие приложения с внешним миром. Этот слой включает:

\begin{itemize}
    \item веб-фреймворки (для рассматриваемой системы~-- \textit{FastAPI}), обрабатывающие \textit{HTTP}-запросы и формирующие ответы;
    \item СУБД и драйверы (\textit{PostgreSQL} через \textit{asyncpg}, \textit{SQLAlchemy Core}), управляющие хранением данных;
    \item брокеры сообщений (для рассматриваемой системы~-- \textit{RabbitMQ}), обеспечивающие асинхронную коммуникацию между сервисами;
    \item сторонние \textit{API} (платежные шлюзы, сервисы аутентификации), интегрируемые через \textit{REST} или \textit{gRPC};
\end{itemize}

Операции пути (роутеры) \textit{FastAPI} выступают точкой входа для внешних запросов. Они выполняют проверку входных данных через \textit{Pydantic}, внедряют зависимости (\textit{Use Cases}, репозитории) через механизм \textit{DI}, преобразуют исключения доменного слоя в \textit{HTTP}-статусы (например, \textit{EmployeeNotFoundError} в \textit{HTTP}-статус 404 с соответствующим сообщением). В листинге ниже представлена операция пути \textit{FastAPI} для обновления данных сотрудника:

\begin{lstlisting}[style=pythonstyle]
@router.patch("/employee/{employee_id}", status_code=status.HTTP_200_OK, response_model=EmployeeOut)
@role_required([EmployeeRoleOption.hr])
async def update_employee(
    employee_id: UUID,
    data: EmployeeUpdate,
    use_case: Annotated[UpdateEmployeeUseCase, Depends(get_update_employee_use_case)],
    employee: Employee = Depends(get_logged_employee),
) -> Employee:
    return await use_case(employee_id, data.model_dump(exclude_unset=True))
\end{lstlisting}

Критерии выбора инструментов включают в себя несколько важных условий, главным из которых является совместимость с асинхронной моделью, а именно поддержка \textit{async/await} (например, использование \textit{asyncpg} вместо \textit{psycopg2} для взаимодействия с базой данных \textit{PostgreSQL}), производительность, безопасность (например, встроенная защита от \textit{SQL}-инъекций в \textit{SQLAlchemy}, \textit{OAuth2} в \textit{FastAPI}~\cite{book_lubanovich_fastapi}).

Таким образом, фреймворки и драйверы, будучи технической реализацией внешнего слоя, обеспечивают эффективное взаимодействие с внешними системами без нарушения границ домена, масштабируемость за счет асинхронности и выделения ресурсов, надёжность через обработку ошибок и транзакции.

Пример потока данных по всем архитектурным слоям логики системы показан на рис.~\ref{fig:system-implementation:software:clean-architecture-data-flow}.

\begin{figure}[h]
\centering
    \includegraphics[width=0.33\linewidth]{assets/clean-architecture-data-flow.png}
    \caption{Схема потоков данных в системе}
    \label{fig:system-implementation:software:clean-architecture-data-flow}
\end{figure}

Подводя итоги, можно отметить, что применение принципов чистой архитектуры в \textit{Python}-проектах обладает рядом значимых преимуществ, хотя и не лишено определённых сложностей. Главным достоинством такого подхода является независимость от сторонних библиотек и фреймворков, что позволяет избежать навязывания их структуры и сохранить гибкость разработки. Компоненты приложения становятся слабосвязанными, что упрощает их замену~-- например, переход на другую базу данных или инструмент требует минимальных изменений, не затрагивая ядро системы. Это также значительно повышает тестируемость: добавление новой функциональности и её проверка становятся проще благодаря чёткому разделению слоёв. Кроме того, проект приобретает прозрачную и масштабируемую структуру, где каждый элемент имеет строго определённую зону ответственности.

Достичь этих результатов помогает комбинация двух ключевых концепций: внедрения зависимостей и инверсии управления. Например, в рассмотренном проекте адаптеры, такие как \textit{EmployeeRepository}, внедряются в сценарии использования, что позволяет абстрагироваться от конкретной реализации. Вместо прямого использования репозитория в бизнес-логике применяются абстрактные классы, задающие интерфейсы для взаимодействия. Это обеспечивает соблюдение границ между слоями архитектуры.

Однако у подхода есть и недостатки. На первый взгляд принципы чистой архитектуры могут показаться сложными из-за необходимости глубокого понимания абстракций и инверсии зависимостей, а структура проекта~-- избыточно громоздкой на первых этапах. Кроме того, декомпозиция и соблюдение строгой изоляции компонентов неизбежно ведут к увеличению объёма кода, что требует дополнительных усилий при разработке. Тем не менее, долгосрочные преимущества в виде поддерживаемости, гибкости и устойчивости к изменениям часто перевешивают эти трудности, особенно в крупных и долгоживущих проектах.


\subsection{Тестирование компонентов системы}
\label{sub:system-implementation:testing}

Тестирование программного обеспечения~-- это ключевая часть процесса разработки, направленная на проверку качества кода, выявление ошибок и обеспечение соответствия системы заданным требованиям. Среди различных видов тестирования можно выделить юнит-тестирование, интеграционное тестирование, регрессионное тестирование и др.

Подход \textit{Test Driven Development (TDD)}~-- это подход, повышающий качество и надежность кода за счет того, что тесты пишутся до написания самого кода, способствуя лучшему проектированию программного обеспечения. Следуя данной методологии, сосредоточимся на желаемом поведении системы, покрывая каждый элемент логики своими тестами.

\textit{\textbf{Юнит-тестирование (Unit Testing)}} фокусируется на проверке отдельных модулей или компонентов системы в изоляции от других частей. Эти тесты предназначены для проверки конкретных функций, методов или классов, чтобы убедиться, что они работают корректно в рамках своих задач. В разрабатываемой системе юнит-тестирование покрывает каждый сценарий использования, проверяя их логику независимо от внешних зависимостей. В листинге ниже приведен пример реализованного \textit{Unit}-теста на языке \textit{Python} с использованием модуля \textit{pytest} для одного из сценариев использования системы.

\begin{lstlisting}[style=pythonstyle]
@pytest.mark.asyncio
async def test_workspace_occupation_request_successfully_created(
    workspace: Workspace,
    user: User,
    request_repository: PostgresWorkspaceOccupationRequestRepository,
    create_request_use_case: CreateWorkspaceOccupationRequestUseCase,
):
    # Preparing request data
    start_time = datetime.now() + timedelta(days=1)
    end_time = start_time + timedelta(hours=2)
    request_data = WorkspaceOccupationRequestData(
        workspace_id=workspace.id,
        user_id=user.id,
        start_time=start_time,
        end_time=end_time,
        status=RequestStatus.PENDING,
    )
    # Creation of an occupation request
    request = await create_request_use_case.execute(request_data)
    # Check that the occupation request is saved in the database
    created_request = await request_repository.get(request.id)
    # Main checks
    assert created_request is not None
    assert created_request.workspace_id == workspace.id
    assert created_request.user_id == user.id
    assert created_request.start_time == start_time
    assert created_request.end_time == end_time
    assert created_request.status == RequestStatus.PENDING
    # Extra checks
    assert created_request.created_at is not None
    assert created_request.updated_at is not None
    assert created_request.updated_at >= created_request.created_at
\end{lstlisting}

\textit{\textbf{Интеграционное тестирование (Integration Testing)}} направлено на проверку взаимодействия между различными компонентами системы. Этот вид тестирования позволяет убедиться, что отдельные модули, уже протестированные на уровне юнит-тестов, корректно работают вместе. В разрабатываемой системе интеграционные тесты покрывают все адаптеры, включая репозитории (работу с \textit{PostgreSQL}) и драйверы (\textit{REST}-интерфейсы \textit{FastAPI}). Это позволяет проверить, как бизнес-логика приложения взаимодействует с базой данных и внешними \textit{API}, обеспечивая надёжность и правильность интеграции с этими компонентами.

Процесс запуска тестов начинается с подготовки тестового окружения, включающего базу данных, фикстуры и зависимости. Тесты выполняются в асинхронном режиме с использованием \textit{\lstinline!pytest.mark.asyncio!}, имитируя реальное выполнение операций. Успешность теста определяется прохождением всех проверок \textit{\lstinline!assert!}, которые включают корректность сохранения данных и наличие обязательных меток времени. В случае неудачи тест завершается с ошибкой, указывающей на первый \textit{\lstinline!assertion!}, завершенный неуспешно. В таблице~\ref{table:system-implementation:testing:booking} представлено пошаговое описание процесса тестирования одного из сценариев использования системы.

\begingroup
\singlespacing
\vspace{-\baselineskip}
\begin{longtable}{|>{\raggedright}m{0.25\textwidth}
                  |>{\raggedright}m{0.46\textwidth}
                  |>{\raggedright\arraybackslash}p{0.2\textwidth}|}
    \caption{Тестирование процесса создания запроса на бронирование рабочего места} \label{table:system-implementation:testing:booking} \\ \hline
    Действие & Ожидаемый результат & Результат теста \\ \hline
    \endfirsthead
    \multicolumn{3}{@{}l}{\noindent Продолжение таблицы~\thetable} \\ \hline
    Действие & Ожидаемый результат & Результат теста \\ \hline
    \endhead
    \multicolumn{3}{|l|}{\textbf{Предусловие}} \\
    \hline
    Переход на страницу, отображающую список офисов & Отображение главного меню с активным пунктом «Офисы», загрузка страницы без ошибок, вывод непустого списка офисов с корректными данными & пройден \\
    \hline
    \multicolumn{3}{|l|}{\textbf{Шаги теста}} \\
    \hline
    Выбор офиса из списка & Отображение страницы с подробной информацией о выбранном офисе, включая список доступных этажей & пройден \\
    \hline
    Выбор этажа & Отображение плана этажа, разметки помещений и расположения рабочих мест с информацией о доступности & пройден \\
    \hline
    Выбор занятого рабочего места & Отображение информации о сотруднике, занявшего выбранное рабочее место. Функционал бронирования скрыт. & пройден \\
    \hline
    Выбор свободного рабочего места & Отображение кнопки «Забронировать» и формы, запрашивающей данные бронирования & пройден \\
    \hline
    Заполнение формы бронирования рабочего места верными данными & Отображение сообщения о том, что запрос на бронирование успешно создан вместе со ссылкой на просмотр деталей запроса & пройден \\
    \hline
    Заполнение формы бронирования рабочего места неверными данными & Отображение сообщения о том, что данные в форме заполнены некорректно. Запрос на бронирование не был создан. & пройден \\
    \hline
\end{longtable}
\endgroup

Каждый вид тестирования имеет свою цель и применяется на определённом этапе разработки. Совместное использование различных видов тестирования обеспечивает качественную проверку как отдельных компонентов, так и их взаимодействия, что делает процесс тестирования неотъемлемой частью создания программного продукта для повышения его надёжности, поддерживаемости и стабильности.



Таким образом, реализация системы охватывает три ключевых направления: аппаратную инфраструктуру, программную архитектуру и тестирование, что обеспечивает комплексный подход к созданию надежного и масштабируемого решения. Аппаратная платформа предполагает использование серверов среднего класса, а инструменты администрирования сети гарантируют стабильность и отказоустойчивость. Сетевая инфраструктура, построенная на управляемых коммутаторах и протоколах (\textit{TCP, TLS, SSH}), обеспечивает безопасную передачу данных. Реализация резервного копирования базы данных \textit{PostgreSQL} минимизирует риски потери данных.

Программная архитектура основана на принципах чистой архитектуры, что обеспечивает четкое разделение слоев (сущности, сценарии использования, адаптеры) и независимость от внешних фреймворков. Это позволяет гибко адаптировать систему под изменения: замену баз данных, интерфейсов или интеграцию новых сервисов. Использование \textit{FastAPI} для \textit{REST}-интерфейсов, асинхронных операций и событийно-ориентированной коммуникации через \textit{RabbitMQ} повышает производительность и масштабируемость. Тестирование играет ключевую роль в обеспечении качества. Реализованные юнит-тесты проверяют корректность отдельных компонентов, а интеграционные тесты охватывают взаимодействие с БД и внешними \textit{API}.

Таким образом, система соответствует современным требованиям к управлению офисными процессами, сочетая экономическую эффективность, масштабируемость и соответствие стандартам качества в разработке ПО.
