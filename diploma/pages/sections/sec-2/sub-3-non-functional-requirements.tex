\subsection{Разработка нефункциональных требований}

Нефункциональные требования описывают атрибуты качества системы, такие как производительность, безопасность, удобство использования и другие характеристики, которые не связаны напрямую с функциональностью, но критически важны для успешной работы системы. Ниже представлены нефункциональные требования для разрабатываемой системы управления офисными информационными процессами.

\begin{enumerate}
    \item \textit{Производительность и масштабируемость}.
        \begin{itemize}
            \item время отклика системы должно составлять не более 2 секунд для 95\% запросов при средней нагрузке и при использовании среднестатистического ADSL Интернет-соединения;
            \item наличие возможностей горизонтального масштабирования для увеличения производительности при росте числа пользователей, что позволит достичь обработки до 10 000 одновременных пользователей без снижения производительности;
            \item нагрузочная устойчивость системы должна обеспечивать полную работоспособность при пиковых нагрузках, например, в начале рабочего дня, когда большинство сотрудников пытаются забронировать рабочие места или оборудование.
        \end{itemize}

    \item \textit{Безопасность}.
        \begin{itemize}
            \item поддержка многофакторной аутентификации для сотрудников. Доступ к функциональности системы должен быть строго регламентирован на основе ролей и прав пользователей. Все личные данные сотрудников, если таковые имеются, должны быть недоступны никому кроме пользователей и администраторов системы;
            \item защита и шифровка передаваемых данных между клиентом и сервером должна быть обеспечена использованием протокола HTTPS. Конфиденциальные данные, такие как биометрическая информация, пароли, и прочая чувствительная ко взлому информация, должны храниться в зашифрованном виде;
            \item журнал всех действий пользователей для последующего аудита должен формироваться в автоматическом режиме для последующей реализации механизмов мониторинга для выявления и предотвращения несанкционированного доступа.
        \end{itemize}

    \item \textit{Удобство использования}.
        \begin{itemize}
            \item интерфейс системы должен быть интуитивно понятным и простым в использовании, с минимальным временем обучения для новых пользователей и поддержкой адаптивного дизайна для корректного отображения на различных устройствах (ПК, планшеты, смартфоны);
            \item наличие возможностей для последующего внедрения поддержки нес\-коль\-ких языков для удобства использования в международных офисах компании;
            \item совместимость с основными операционными системами (\textit{Win\-dows}, \textit{macOS}, \textit{Linux}) и браузерами (\textit{Chrome}, \textit{Firefox}, \textit{Safari}, \textit{Edge}).
        \end{itemize}

    \item \textit{Надежность и отказоустойчивость}.
        \begin{itemize}
            \item доступность системы должна составлять минимум 95\% в течение года;
            \item регулярное резервное копирование данных должно выполняться в автоматическом режиме для обеспечения возможности восстановления в течение 1 часа после сбоя;
        \end{itemize}

    \item \textit{Техническая поддержка и обслуживание}.
        \begin{itemize}
            \item система должна поддерживать возможность обновления без прерывания работы пользователей;
            \item полная техническая и пользовательская документация должна быть доступна для всех модулей системы, а также должна регулярно обновляться в соответствии с изменениями.
        \end{itemize}
\end{enumerate}
