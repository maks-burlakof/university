\sectioncentered*{Введение}
\addcontentsline{toc}{section}{Введение}
\label{sec:introduction}

В условиях современного бизнеса, где компании активно расширяют свою географию и внедряют гибридные форматы работы, эффективное управление офисными ресурсами становится критически важным. Сотрудники работают в различных локациях, а офисные процессы, такие как бронирование рабочих мест, оборудования и помещений, требуют автоматизации для повышения производительности и минимизации ручного труда. Главным преимуществом является высвобождение человеческих ресурсов и направление потенциала сотрудников на выполнение других, более значимых и стратегических поручений. Разрабатываемая система управления офисными информационными процессами спроектирована для решения этих задач, предоставляя единую платформу для автоматизации рабочих процессов, управления доступом и интеграции с техническими средствами контроля.

Основной целью системы является создание удобного и автоматизированного инструмента для сотрудников и менеджеров, который позволит эффективно управлять офисными ресурсами, такими как рабочие места, оборудование и помещения общего пользования. Система обеспечивает автоматическую рассадку сотрудников, контроль доступа в помещения на основе ролей и прав, а также интеграцию с техническими средствами, такими как считыватели карт и биометрические системы. Это позволяет минимизировать ручное вмешательство, сократить человеческий фактор и повысить уровень автоматизации рутинных офисных процессов. Автоматическая генерация отчетности позволяет мгновенно получить информацию, необходимую для принятия разумных решений, обеспечивающих экономию средств и повышение эффективности.

Актуальность разработки такой системы обусловлена растущими потребностями компаний в оптимизации использования офисных ресурсов, особенно в условиях гибридной работы, когда сотрудники частично работают удаленно, а частично~-- в офисе. Система не только упрощает процессы бронирования и управления доступом, но и предоставляет аналитические инструменты для оценки эффективности использования ресурсов, что помогает принимать обоснованные управленческие решения. Использование системы для разработки стратегии использования имеющихся площадей, составления планов этажей и схем рассадки поможет выявить недостаточно используемые помещения и в конечном итоге приведет к экономии средств. Востребованность таких систем для автоматизации способствует постоянному развитию этой сферы, программные продукты становятся более функциональными и удобными, чтобы удовлетворить растущие запросы бизнеса.

В данном дипломном проекте рассматриваются ключевые аспекты разработки системы, включая анализ требований, проектирование архитектуры, выбор технологий и реализацию функциональных модулей. Особое внимание уделяется автоматизации процессов бронирования, управления доступом и интеграции с внешними техническими системами и устройствами. Результатом проекта станет готовое решение, которое может быть внедрено в компании для повышения эффективности управления офисными процессами и улучшения условий работы сотрудников.

Таким образом, разрабатываемая система представляет собой современное и гибкое решение, которое отвечает потребностям компаний в автоматизации офисных процессов, обеспечивая удобство, безопасность и высокую производительность.

Данный дипломный проект выполнен мной лично, проверен на заимствования, процент оригинальности соответствует требованиям, установленными кафедрой.
