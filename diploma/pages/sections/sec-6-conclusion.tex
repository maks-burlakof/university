\sectioncentered*{Заключение}
\addcontentsline{toc}{section}{Заключение}

В ходе выполнения дипломного проекта была разработана система управления офисными информационными процессами, направленная на автоматизацию офисной инфраструктуры, оптимизацию использования ресурсов и повышение эффективности взаимодействия сотрудников.

В первом разделе был проведён анализ предметной области, в рамках которого рассмотрены ключевые процессы офисной среды: бронирование рабочих мест и оборудования, управление доступом, автоматизация рутинных операций. Были изучены существующие решения, такие как \textit{MRI Workplace Central} и \textit{Envoy}, проведён сравнительный анализ подходов к автоматизации офисных процессов, что позволило выявить наиболее эффективные практики и требования к разрабатываемой системе.

Во втором разделе определены функциональные и нефункциональные требования к системе. Сформированы сценарии использования, охватывающие бронирование ресурсов, автоматическую рассадку сотрудников, контроль доступа, управление оборудованием и отчётность. Уделено внимание вопросам безопасности, производительности и масштабируемости системы, а также удобству использования. Были спроектированы модули с чётким разделением ролей и автоматизированными механизмами согласования.

В третьем разделе выполнено проектирование архитектуры системы. В качестве архитектурного шаблона выбрана микросервисная модель, сочетающая преимущества клиент-серверной архитектуры с возможностями масштабирования и независимого развития компонентов. Разработаны модели данных с использованием реляционных СУБД, определены алгоритмы работы ключевых модулей, продумана схема взаимодействия сервисов через брокер сообщений. Особое внимание уделено развёртыванию в контейнеризованной среде с использованием \textit{Docker} и \textit{Kubernetes}.

В четвёртом разделе описана реализация системы. Аппаратная инфраструктура построена на базе серверов среднего класса, с использованием управляемых сетевых компонентов и протоколов безопасной передачи данных. Серверная часть реализована на языке \textit{Python} с использованием фреймворка \textit{FastAPI}, клиентская~-- на \textit{React} с \textit{TypeScript}. Организовано резервное копирование баз данных \textit{PostgreSQL}. Проведены модульные и интеграционные тесты, подтверждающие корректность функционирования компонентов.

В пятом разделе выполнено технико-экономическое обоснование разработки системы. Расчёт затрат показал целесообразность создания собственного решения. Вычислен экономический эффект от внедрения, связанный с сокращением трудозатрат, повышением прозрачности процессов и эффективным использованием ресурсов. Расчёт показателей рентабельности подтвердил эффективность и востребованность предложенного программного продукта.

Таким образом, в процессе дипломного проектирования была разработана гибкая, модульная и масштабируемая система управления офисными информационными процессами, соответствующая современным требованиям. Система обеспечивает автоматизацию ключевых бизнес-процессов, улучшает условия работы сотрудников, способствует снижению операционных издержек и повышает конкурентоспособность организации. Возможность интеграции с внешними сервисами, архитектурная устойчивость к росту нагрузки и ориентация на аналитическую отчётность делают систему перспективной и удобной для внедрения в условиях распределённой корпоративной среды.
