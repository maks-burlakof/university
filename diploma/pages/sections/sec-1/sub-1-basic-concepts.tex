\subsection{Основные понятия и определения}
\label{sub:domain:basic-concepts}

Офисные помещения представляют собой специально оборудованные территории, предназначенные для организации и ведения управленческой, административной и творческой деятельности, где взаимодействуют как материальные, так и нематериальные активы предприятия. Эти пространства характеризуются высоким уровнем функциональной гибкости и интеграции, позволяющей эффективно сочетать традиционные методы работы с современными информационными и коммуникационными технологиями. В данном контексте офис выступает не только как физическое место расположения сотрудников, но и как система, объединяющая в себе ресурсы, такие как рабочие места, конференц-залы, переговорные комнаты, зоны отдыха, а также вспомогательные сервисы, инфраструктуру информационных технологий и средства безопасности.

В офисных помещениях происходит целый ряд процессов, составляющих сложную информационно-территориальную систему и направленных на обеспечение непрерывности бизнес-операций, оптимизацию использования ресурсов и повышение производительности труда. Среди них можно выделить процессы бронирования рабочих мест, позволяющие сотрудникам оперативно резервировать конкретные зоны для выполнения своих задач, а также организацию гибкого графика использования офисных территорий, что способствует адаптации к изменяющимся требованиям современного рынка~\cite{web_automatization_of_operational_processes_in_company}. Помимо этого, значимым является процесс распределения и управления материальными активами, который включает в себя инвентаризацию офисной техники, мебели, коммуникационного оборудования и других материальных ценностей, обеспечивая контроль над их состоянием, перемещением и износом.

Другой важный аспект~-- это организация коммуникационных процессов, обеспечивающих эффективное взаимодействие между сотрудниками посредством систем видеоконференций, коллективных чатов и корпоративных порталов, что способствует формированию единой информационной среды. Кроме того, процессы технического обслуживания и модернизации инфраструктуры играют критическую роль в поддержании рабочих условий на должном уровне, включают в себя регулярное проведение профилактических работ, ремонт оборудования, а также модернизацию систем безопасности, в том числе контроля доступа.

В совокупности все перечисленные процессы образуют взаимосвязанную систему, в которой материальные и нематериальные активы интегрируются для обеспечения эффективного функционирования компании, повышая её конкурентоспособность и адаптивность в условиях динамичных изменений внешней среды.

В основе офисных информационных систем лежит глубокая автоматизация бизнес-процессов, начиная с бронирования рабочих мест, общих помещений~-- коворкингов, что позволяет сотрудникам самостоятельно подавать запросы на временное или постоянное использование конкретных позиций в офисе, а также управлять их распределением с учетом планировок помещений, этажей и отдельных зон. Эта процедура, интегрированная с модулем анализа текущей загрузки пространства, позволяет обеспечить адаптивное распределение ресурсов в зависимости от интенсивности использования и оперативных потребностей компании. Наряду с этим, процесс бронирования оборудования, который учитывает не только наличие техники в разных локациях, но и дифференцированный доступ, основанный на ролях и наборе разрешений сотрудников, позволяет реализовать гибкую политику управления материальными активами. 

Внедрение алгоритмов автоматической рассадки позволяет на основе заданных параметров, таких как специализация, географическая привязанность и текущая нагрузка офисных зон, проводить оптимальное распределение сотрудников, что минимизирует операционные издержки и повышает продуктивность.

Автоматизированный контроль доступа позволяет формировать единый механизм аутентификации и авторизации пользователей в различных зонах. Он реализуется посредством интеграции с техническими средствами идентификации, включая персональные карточки и биометрические устройства. Этот комплекс мер не только упрощает процедуру выдачи прав доступа и контроля за перемещением сотрудников, но и способствует точному учету времени нахождения, что является важным показателем операционной эффективности.
