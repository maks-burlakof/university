% !TeX spellcheck = russian-aot-ieyo

% Config file for overleaf
% System: TeXLive version 2023; pdfLaTeX compiler

% Требования СТП: https://www.bsuir.by/m/12_100229_1_122976.pdf

% ---------------------------- 1 FONTS ----------------------------

\documentclass[a4paper,14pt,russian,oneside,final]{report}
\usepackage[utf8]{inputenc}

% Чтобы можно было использовать русские буквы в формулах, но в случае использования предупреждать об этом.
\usepackage[warn]{mathtext}

\usepackage[english,main=russian]{babel}
\usepackage[T1,T2A]{fontenc}
\usefont{T2A}{ftm}{m}{sl}

% make PDF "searchable and copyable".
\usepackage{cmap}

% Курсив и жирность для кириллицы
\usepackage{substitutefont}

\substitutefont{T2A}{\familydefault}{Tempora-TLF}
\makeatletter
\input{t2atempora-tlf.fd}
\DeclareFontShape{T2A}{Tempora-TLF}{m}{sc}{
         <-> ssub * Tempora-TLF/m/n
}{}

% Зачем: Добавляет поддержу дополнительных размеров текста 8pt, 9pt, 10pt, 11pt, 12pt, 14pt, 17pt, and 20pt.
% Почему: Пункт 2.1.1 Требований по оформлению пояснительной записки.
\usepackage{extsizes}

% Зачем: Длина, примерно соответвующая 5 символам
% Почему: Требования содержат странное требование про отсупы в 5 символов (для немоноширинного шрифта :| )
\newlength{\fivecharsapprox}
\setlength{\fivecharsapprox}{6ex}

% Зачем: Выбор шрифта по умолчанию. 
% Почему: Пункт 2.1.1 Требований по оформлению пояснительной записки.
% Примечание: В требованиях не указан, какой именно шрифт использовать. По традиции используем TNR.
% \renewcommand{\rmdefault}{ftm} % Times New Roman

% Для подчеркнутого текста используя \uline
\usepackage[normalem]{ulem}
% \renewcommand{\ULthickness}{0.8pt}   % Толщина линии
\setlength{\ULdepth}{0.12em}          % Расстояние от текста до линии

\usepackage[square,numbers,sort&compress]{natbib}
\setlength{\bibsep}{0em}

\PassOptionsToPackage{hyphens}{url}

\usepackage{hyphenat}

% Зачем: Поддержка ажурного и готического шрифтов 
\usepackage{amsfonts}

% Зачем: amsfonts + несколько сотен дополнительных математических символов
\usepackage{amssymb}

% Зачем: Задает стиль заголовков раздела жирным шрифтом, прописными буквами, без точки в конце
% Почему: Пункты 2.1.1, 2.2.5, 2.2.6 и ПРИЛОЖЕНИЕ Л Требований по оформлению пояснительной записки.
\makeatletter
\renewcommand\section{%
  \clearpage\@startsection {section}{1}%
    {\fivecharsapprox}%
    {-1em \@plus -1ex \@minus -.2ex}%
    {1em \@plus .2ex}%
    {\raggedright\hyphenpenalty=10000\normalfont\large\bfseries\MakeUppercase}}
\makeatother

% Зачем: Задает стиль заголовков подразделов
% Почему: Пункты 2.1.1, 2.2.5 и ПРИЛОЖЕНИЕ Л Требований по оформлению пояснительной записки.
\makeatletter
\renewcommand\subsection{%
  \@startsection{subsection}{2}%
    {\fivecharsapprox}%
    {-1em \@plus -1ex \@minus -.2ex}%
    {1em \@plus .2ex}%
    {\raggedright\hyphenpenalty=10000\normalfont\normalsize\bfseries}}
\makeatother

% Зачем: Задает стиль заголовков пунктов
% Почему: Пункты 2.1.1, 2.2.5 и ПРИЛОЖЕНИЕ Л Требований по оформлению пояснительной записки.
\makeatletter
\renewcommand\subsubsection{
  \@startsection{subsubsection}{3}%
    {\fivecharsapprox}%
    {-1em \@plus -1ex \@minus -.2ex}%
    {\z@}%
    {\raggedright\hyphenpenalty=10000\normalfont\normalsize\bfseries}}
\makeatother

% Зачем: для оформления введения и заключения, они должны быть выровнены по центру.
% Почему: Пункты 1.1.15 и 1.1.11 Требований по оформлению пояснительной записки.
\makeatletter
\newcommand\sectioncentered{%
  \clearpage\@startsection {section}{1}%
    {\z@}%
    {-1em \@plus -1ex \@minus -.2ex}%
    {1em \@plus .2ex}%
    {\centering\hyphenpenalty=10000\normalfont\large\bfseries\MakeUppercase}%
    }
\makeatother

% CONFVAR: belarus-specific-utf8gost780u
% Зачем: Задает стиль библиографии
% Почему: Пункт 2.8.6 Требований по оформлению пояснительной записки.
\bibliographystyle{confs/styles/belarus-specific-utf8gost780u}



% --------------------------- 2 SPACING ---------------------------

% 2.1.3 отступы для абзацев
\usepackage{indentfirst}
\setlength{\parindent}{\fivecharsapprox} % Примерно соответсвует 5 символам.

% 2.1.2: отступы от границ страницы
\usepackage[top=20mm, bottom=20mm, left=30mm, right=15mm]{geometry}

% 2.1.1: межстрочный интервал для размещения 40 +/- 3 строки текста на странице
\usepackage[nodisplayskipstretch]{setspace} 
\setstretch{1.1}
%\onehalfspacing

% 2.2.7: отступ между таблицей с содержанимем и словом СОДЕРЖАНИЕ
\usepackage{tocloft}
\setlength{\cftbeforetoctitleskip}{-1em}
\setlength{\cftaftertoctitleskip}{1em}

% 2.2.7: отступы слева для записей в таблице содержания.
\makeatletter
\renewcommand{\l@section}{\@dottedtocline{1}{0.5em}{1.2em}}
\renewcommand{\l@subsection}{\@dottedtocline{2}{1.7em}{2.0em}}
\makeatother

% отключает использование изменяемых межсловных пробелов
% так не принято делать в текстах на русском языке
\frenchspacing

% Добавляем абзацный отступ для библиографии
% https://github.com/mstyura/bsuir-diploma-latex/issues/19
\setlength\bibindent{-1.0900cm}

\makeatletter
\renewcommand\NAT@bibsetnum[1]{\settowidth\labelwidth{\@biblabel{#1}}%
   \setlength{\leftmargin}{\bibindent}\addtolength{\leftmargin}{\dimexpr\labelwidth+\labelsep\relax}%
   \setlength{\itemindent}{-\bibindent+\fivecharsapprox-0.240cm}%
   \setlength{\listparindent}{\itemindent}
\setlength{\itemsep}{\bibsep}\setlength{\parsep}{\z@}%
   \ifNAT@openbib
     \addtolength{\leftmargin}{\bibindent}%
     \setlength{\itemindent}{-\bibindent}%
     \setlength{\listparindent}{\itemindent}%
     \setlength{\parsep}{10pt}%
   \fi
}



% -------------------------- 3 СНОСКИ --------------------------

% Зачем: Сброс счетчика сносок для каждой страницы
% Примечание: в "Требованиях по оформлению пояснительной записки" не указано, как нужно делать, но в других БГУИРовских докуметах рекомендуется нумерация отдельная для каждой страницы
\usepackage{perpage}
\MakePerPage{footnote}

% 2.9.2, 2.9.1: добавляет скобку 1) к номеру сноски
\makeatletter 
\def\@makefnmark{\hbox{\@textsuperscript{\normalfont\@thefnmark)}}}
\makeatother

% 2.9.2: расположение сносок внизу страницы
\usepackage[bottom]{footmisc}

% Зачем: Переопределяем стандартную нумерацию, т.к. в отчете будут только section и т.д. в терминологии TeX
\makeatletter
\renewcommand{\thesection}{\arabic{section}}
\makeatother

% 2.2.3: пункты (в терминологии требований) в терминологии TeX subsubsection должны нумероваться
\setcounter{secnumdepth}{3}

% Зачем: Работа с колонтитулами
\usepackage{fancyhdr} % пакет для установки колонтитулов
\pagestyle{fancy} % смена стиля оформления страниц

% 2.2.8: нумерация страниц располагается справа снизу страницы
\fancyhf{} % очистка текущих значений
\fancyfoot[R]{\fontsize{14}{16}\selectfont \thepage}
% \fancyfoot[R]{\thepage} % установка верхнего колонтитула
\renewcommand{\footrulewidth}{0pt} % убрать разделительную линию внизу страницы
\renewcommand{\headrulewidth}{0pt} % убрать разделительную линию вверху страницы
\fancypagestyle{plain}{ 
    \fancyhf{}
    \rfoot{\thepage}}



% -------------------------- 4 FIGURES --------------------------

% Зачем: Пакет для вставки картинок
% Примечание: Объяснение, зачем final - http://tex.stackexchange.com/questions/11004/why-does-the-image-not-appear
\usepackage[final]{graphicx}
\DeclareGraphicsExtensions{.pdf,.png,.jpg,.eps}

% CONFVAR: assets
% Зачем: Директория в которой будет происходить поиск картинок
\graphicspath{{assets/}}

% Зачем: Добавление подписей к рисункам
\usepackage[nooneline]{caption}
\usepackage{subcaption}

% Зачем: чтобы работала \No в новых латехах
\DeclareRobustCommand{\No}{\ifmmode{\nfss@text{\textnumero}}\else\textnumero\fi}

% Зачем: поворот ячеек таблиц на 90 градусов
\usepackage{rotating}
\DeclareRobustCommand{\povernut}[1]{\begin{sideways}{#1}\end{sideways}}

% Зачем: Задание подписей, разделителя и нумерации частей рисунков
% Почему: Пункт 2.5.5 Требований по оформлению пояснительной записки.
\DeclareCaptionLabelFormat{stbfigure}{Рисунок #2}
\DeclareCaptionLabelFormat{stbtable}{Таблица #2}
\DeclareCaptionLabelSeparator{stb}{~--~}
\captionsetup{labelsep=stb}

%\captionsetup[figure]{labelformat=stbfigure,justification=centering}
\captionsetup[figure]{labelformat=stbfigure,justification=centering,skip=15pt}
%\captionsetup[table]{labelformat=stbtable,justification=raggedright}
\captionsetup[table]{labelformat=stbtable,justification=raggedright,format=hang,skip=0pt}
%\captionsetup[subfigure]{labelformat=stbsubfigure,labelsep=stbsubfiguresep}
\renewcommand{\thesubfigure}{\asbuk{subfigure}}

% Зачем: Включение номера раздела в номер формулы. Нумерация формул внутри раздела.
\AtBeginDocument{\numberwithin{equation}{section}}

% Зачем: Включение номера раздела в номер таблицы. Нумерация таблиц внутри раздела.
\AtBeginDocument{\numberwithin{table}{section}}

% Зачем: Включение номера раздела в номер рисунка. Нумерация рисунков внутри раздела.
\AtBeginDocument{\numberwithin{figure}{section}}

% Зачем: Дополнительные возможности в форматировании таблиц
\usepackage{makecell}
\usepackage{multirow}
\usepackage{array}
\usepackage{ragged2e}

% Для оформления таблиц не влязящих на 1 страницу
\usepackage{longtable}



% -------------------------- 4 FORMULAS --------------------------

% Зачем: когда в формулах много кириллических символов команда \text{} занимает много места
\DeclareRobustCommand{\x}[1]{\text{#1}}

% Зачем: Окружения для оформления формул
% Почему: Пункт 2.4.7 требований по оформлению пояснительной записки и специфические требования различных кафедр
% Пример использования смотри в course_content.tex, строка 5
\usepackage{calc}
\newlength{\lengthWordWhere}
\settowidth{\lengthWordWhere}{где}
\newenvironment{explanationx}
    {%
    %%% Следующие строки определяют специфические требования разных редакций стандартов. Раскоменнтируйте нужную строку
    %% стандартный абзац, СТП-01 2010
    %\begin{itemize}[leftmargin=0cm, itemindent=\parindent + \lengthWordWhere + \labelsep, labelsep=\labelsep]
    %% без отступа, СТП-01 2013
    \begin{itemize}[leftmargin=0cm, itemindent=\lengthWordWhere + \labelsep , labelsep=\labelsep]%
    \renewcommand\labelitemi{}%
    }
    {%
    %\\[\parsep]
    \end{itemize}
    }

% Старое окружение для "где". Сохранено для совместимости
\usepackage{tabularx}

\newenvironment{explanation}
    {
    %%% Следующие строки определяют специфические требования разных редакций стандартов. Раскоменнтируйте нужные 2 строки
    %% стандартный абзац, СТП-01 2010
    %\par 
    %\tabularx{\textwidth-\fivecharsapprox}{@{}ll@{ --- } X }
    %% без отступа, СТП-01 2013
    \noindent 
    \tabularx{\textwidth}{@{}ll@{ --- } X }
    }
    { 
    \\[\parsep]
    \endtabularx
    }

% Зачем: Удобная вёрстка многострочных формул, масштабирующийся текст в формулах, формулы в рамках и др
\usepackage{amsmath}

% Зачем: Окружения «теорема», «лемма»
\usepackage{amsthm}

% Зачем: Производить арифметические операции во время компиляции TeX файла
\usepackage{calc}

% Зачем: Производить арифметические операции во время компиляции TeX файла
\usepackage{fp}

% Зачем: "Умная" запятая в математических формулах. В дробных числах не добавляет пробел
% Почему: В требованиях не нашел, но в русском языке для дробных чисел используется {,} а не {.}
\usepackage{icomma}

% Зачем: макрос для печати римских чисел
\makeatletter
\newcommand{\rmnum}[1]{\romannumeral #1}
\newcommand{\Rmnum}[1]{\expandafter\@slowromancap\romannumeral #1@}
\makeatother

% Зачем: Управление выводом чисел.
\usepackage{sistyle}
\SIdecimalsign{,}



% --------------------------- 5 LISTS ---------------------------

% Зачем: Пакет для работы с перечислениями
\usepackage{enumitem}
\makeatletter
 \AddEnumerateCounter{\asbuk}{\@asbuk}{щ)}
\makeatother

% Зачем: Устанавливает символ начала простого перечисления
% Почему: Пункт 2.3.5 Требований по оформлению пояснительной записки.
\setlist{nolistsep}

% Зачем: Устанавливает символ начала именованного перечисления
% Почему: Пункт 2.3.8 Требований по оформлению пояснительной записки.
\renewcommand{\labelenumi}{\asbuk{enumi})}
\renewcommand{\labelenumii}{\arabic{enumii})}

% Зачем: Устанавливает отступ от границы документа до символа списка, чтобы этот отступ равнялся отступу параграфа
% Почему: Пункт 2.3.5 Требований по оформлению пояснительной записки.

\setlist[itemize,0]{itemindent=\parindent + 2.2ex,leftmargin=0ex,label=--}
\setlist[enumerate,1]{itemindent=\parindent + 2.7ex,leftmargin=0ex}
\setlist[enumerate,2]{itemindent=\parindent + \parindent - 2.7ex}



% ---------------------------- 6 CODE ----------------------------

% Зачем: inline-коментирование содержимого.
\newcommand{\ignore}[2]{\hspace{0in}#2}

% Зачем: Возможность коментировать большие участки документа
\usepackage{verbatim}

\usepackage{xcolor}

% Зачем: Оформление листингов кода
% Примечание: final нужен для переопределения режима draft, в котором листинги не выводятся в документ.
\usepackage[final]{listings}

% Зачем: настройка оформления листинга
\definecolor{bluekeywords}{rgb}{0.13,0.13,1}
\definecolor{greencomments}{rgb}{0,0.5,0}
\definecolor{turqusnumbers}{rgb}{0.17,0.57,0.69}
\definecolor{redstrings}{rgb}{0.5,0,0}

\renewcommand{\lstlistingname}{Листинг}

\lstdefinelanguage{JavaScript}{
    morekeywords={
        break,case,catch,class,const,continue,debugger,default,delete,do,else,export,extends,finally,for,function,if,import,in,instanceof,let,new,return,super,switch,this,throw,try,typeof,var,void,while,with,yield,async,await
    },
    keywordstyle=\bfseries\color{bluekeywords},
    sensitive=true,
    morecomment=[l][\color{greencomments}]{//},
    morecomment=[s][\color{greencomments}]{/*}{*/},
    morestring=[b]",
    morestring=[b]',
    stringstyle=\color{redstrings},
}

\lstdefinestyle{jsstyle}{
    xleftmargin=0ex,
    language=JavaScript,
    basicstyle=\footnotesize\ttfamily,
    breaklines=true,
    columns=fullflexible,
}

\lstdefinestyle{jsinlinestyle} {
    language=JavaScript,
    breaklines=true,
    columns=fullflexible,
    basicstyle=\footnotesize\ttfamily,
}

\lstdefinestyle{jsframestyle}{
    language=JavaScript,
    frame=lr,
    rulecolor=\color{blue!80!black},
}

\lstdefinelanguage{Python}{
    morekeywords={
        False,None,True,and,as,assert,async,await,break,class,continue,def,del,elif,
        else,except,finally,for,from,global,if,import,in,is,lambada,nonlocal,not,or,
        pass,raise,return,try,while,with,yield
    },
    keywordstyle=\bfseries\color{bluekeywords},
    sensitive=true,
    morecomment=[l][\color{greencomments}]{\#},
    morestring=[b]",
    morestring=[s]{'''}{'''},
    morestring=[s]{"""}{"""},
    stringstyle=\color{redstrings},
}

\lstdefinestyle{pythonstyle}{
    xleftmargin=0ex,
    language=Python,
    basicstyle=\footnotesize\ttfamily,
    breaklines=true,
    columns=fullflexible,
}

\lstdefinestyle{pythoninlinestyle} {
    language=Python,
    breaklines=true,
    columns=fullflexible,
    basicstyle=\footnotesize\ttfamily,
}

\lstdefinestyle{pythonframestyle}{
    language=Python,
    frame=lr,
    rulecolor=\color{blue!80!black},
}

\lstdefinelanguage{C}{
    morekeywords={
        auto,break,case,char,const,continue,default,do,double,else,enum,extern,float,
        for,goto,if,inline,int,long,register,restrict,return,short,signed,sizeof,static,
        struct,switch,typedef,union,unsigned,void,volatile,while,_Alignas,_Alignof,_Atomic,
        _Bool,_Complex,_Decimal128,_Decimal64,_Decimal32,_Generic,_Imaginary,_Noreturn,_Static_assert,
        _Thread_local
    },
    keywordstyle=\bfseries\color{bluekeywords},
    sensitive=true,
    morecomment=[l][\color{greencomments}]{//},
    morecomment=[s][\color{greencomments}]{/*}{*/},
    morestring=[b]",
    morestring=[b]',
    stringstyle=\color{redstrings},
}

\lstdefinestyle{cstyle}{
    xleftmargin=0ex,
    language=C,
    basicstyle=\footnotesize\ttfamily,
    breaklines=true,
    columns=fullflexible,
}

\lstdefinestyle{cinlinestyle} {
    language=C,
    breaklines=true,
    columns=fullflexible,
    basicstyle=\footnotesize\ttfamily,
}

\lstdefinestyle{cframestyle}{
    language=C,
    frame=lr,
    rulecolor=\color{blue!80!black},
}

\lstdefinelanguage{FSharp}
    {morekeywords={abstract,and,as,assert,base,begin,class,default,delegate,do,done,downcast,downto,elif,else,end,exception,extern,false,finally,for,fun,function,global,if,in,inherit,inline,interface,internal,lazy,let,let!,match,member,module,mutable,namespace,new,not,null,of,open,or,override,private,public,rec,return,return!,select,static,struct,then,to,true,try,type,upcast,use,use!,val,void,when,while,with,yield,yield!,asr,land,lor,lsl,lsr,lxor,mod,sig,atomic,break,checked,component,const,constraint,constructor,continue,eager,event,external,fixed,functor,include,method,mixin,object,parallel,process,protected,pure,sealed,tailcall,trait,virtual,volatile},
    keywordstyle=\bfseries\color{bluekeywords},
    sensitive=false,
    morecomment=[l][\color{greencomments}]{///},
    morecomment=[l][\color{greencomments}]{//},
    morecomment=[s][\color{greencomments}]{{(*}{*)}},
    morestring=[b]",
    stringstyle=\color{redstrings},
    }

\lstdefinestyle{fsharpstyle}{
   xleftmargin=0ex,
   language=FSharp,
   basicstyle=\footnotesize\ttfamily,
   breaklines=true,
   columns=fullflexible
}

\lstdefinestyle{csharpinlinestyle} {
  language=[Sharp]C,
  morekeywords={yield,var,get,set,from,select,partial,where,async,await},
  breaklines=true,
  columns=fullflexible,
  basicstyle=\footnotesize\ttfamily
}

\lstdefinestyle{csharpstyle}{
  language=[Sharp]C,
  frame=lr,
  rulecolor=\color{blue!80!black}
}


% Зачем: Нумерация листингов в пределах секции
\AtBeginDocument{\numberwithin{lstlisting}{section}}

\usepackage[final,hidelinks]{hyperref}
% Моноширинный шрифт выглядит визуально больше, чем пропорциональный шрифт, если их размеры одинаковы. Искусственно уменьшаем размер ссылок.
\renewcommand{\UrlFont}{\small\rmfamily\tt}



% ---------------------------- OTHER ----------------------------

% Для включения pdf документов в результирующий файл
\usepackage{pdfpages}

% Для альбомных страниц в PDF
\usepackage{pdflscape}
\usepackage{afterpage}

% Для использования знака градуса и других знаков
% http://ctan.org/pkg/gensymb
\usepackage{gensymb}

% Зачем: преобразовывать текст в верхний регистр командой MakeTextUppercase
\usepackage{textcase}

% Зачем: Переносы в словах с тире.
% Тире в словае заменяем на \hyph: аппаратно\hyphпрограммный.
% https://stackoverflow.com/questions/2193307/how-to-get-latex-to-hyphenate-a-word-that-contains-a-dash#
\def\hyph{-\penalty0\hskip0pt\relax}


% Магия для подсчета разнообразных объектов в документе
\usepackage{lastpage}
\usepackage{totcount}
\regtotcounter{section}

\usepackage{etoolbox}

\newcounter{totfigures}
\newcounter{tottables}
\newcounter{totreferences}
\newcounter{totequation}

\providecommand\totfig{} 
\providecommand\tottab{}
\providecommand\totref{}
\providecommand\toteq{}

\makeatletter
\AtEndDocument{%
  \addtocounter{totfigures}{\value{figure}}%
  \addtocounter{tottables}{\value{table}}%
  \addtocounter{totequation}{\value{equation}}
  \immediate\write\@mainaux{%
    \string\gdef\string\totfig{\number\value{totfigures}}%
    \string\gdef\string\tottab{\number\value{tottables}}%
    \string\gdef\string\totref{\number\value{totreferences}}%
    \string\gdef\string\toteq{\number\value{totequation}}%
  }%
}
\makeatother

\pretocmd{\section}{\addtocounter{totfigures}{\value{figure}}\setcounter{figure}{0}}{}{}
\pretocmd{\section}{\addtocounter{tottables}{\value{table}}\setcounter{table}{0}}{}{}
\pretocmd{\section}{\addtocounter{totequation}{\value{equation}}\setcounter{equation}{0}}{}{}
\pretocmd{\bibitem}{\addtocounter{totreferences}{1}}{}{}
